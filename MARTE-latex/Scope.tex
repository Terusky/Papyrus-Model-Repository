\section{Scope}

\subsection{Introduction}

% This specification of a UML™ profile adds capabilities to UML for model-driven development of Real Time and Embedded Systems (RTES). 
% This extension, called the UML profile for MARTE (in short MARTE for Modeling and Analysis of Real-Time and Embedded systems), provides support for specification, design, and verification/validation
% stages. 
% This new profile is intended to replace the existing UML Profile for Schedulability, Performance and Time (formal/03-09-01).

UML™プロファイルのこの仕様は、リアルタイムおよび組み込みシステム(RTES)のモデル駆動型開発のための機能をUMLに追加します。
MARTEのUMLプロファイル(略して、リアルタイムおよび組み込みシステムのモデリングおよび分析のMARTE)と呼ばれるこの拡張機能は、仕様、設計、および検証/検証のサポートを提供します
ステージ。
この新しいプロファイルは、スケジュール可能性、パフォーマンス、および時間(formal / 03-09-01)の既存のUMLプロファイルを置き換えることを目的としています。

% MARTE defines foundations for model-based descriptions of real time and embedded systems. 
% These core concepts are then refined for both modeling and analyzing concerns. 
% Modeling parts provides support required from specification to detailed design of real-time and embedded characteristics of systems. 
% MARTE concerns also model-based analysis. 
% In this sense, the intent is not to define new techniques for analyzing real-time and embedded systems, but to support them.
% Hence, it provides facilities to annotate models with information required to perform specific analysis. 
% Especially, MARTE focuses on performance and schedulability analysis. But, it defines also a general analysis framework that intends to refine/specialize any other kind of analysis.

MARTEは、リアルタイムおよび組み込みシステムのモデルベースの記述の基礎を定義します。
これらのコアコンセプトは、問題のモデリングと分析の両方のために洗練されます。
モデリングパーツは、システムのリアルタイムおよび組み込み特性の仕様から詳細設計に必要なサポートを提供します。
MARTEはモデルベースの分析にも関係しています。
この意味での意図は、リアルタイムおよび組み込みシステムを分析するための新しい手法を定義することではなく、それらをサポートすることです。
したがって、特定の分析を実行するために必要な情報でモデルに注釈を付ける機能を提供します。
特に、MARTEはパフォーマンスとスケジュール可能性の分析に重点を置いています。ただし、他の種類の分析を改良/専門化することを目的とした一般的な分析フレームワークも定義します。

%Among others, the benefits of using this profile are thus:
特に、このプロファイルを使用する利点は次のとおりです。
\begin{itemize}
%   \item  Providing a common way of modeling both hardware and software aspects of an RTES in order to improve communication between developers.
%   \item  Enabling interoperability between development tools used for specification, design, verification, code generation, etc.
%   \item  Fostering the construction of models that may be used to make quantitative predictions regardin
  \item  開発者間のコミュニケーションを改善するために、RTESのハードウェアとソフトウェアの両方の側面をモデル化する一般的な方法を提供します。
  \item  仕様、設計、検証、コード生成などに使用される開発ツール間の相互運用性を可能にします。
  \item  定量的な予測を行うために使用できるモデルの構築を促進する
\end{itemize}
